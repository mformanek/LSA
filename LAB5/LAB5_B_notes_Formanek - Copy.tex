\documentclass[11pt,onside]{article}
\usepackage[a4paper]{geometry}
\usepackage[utf8]{inputenc}
\usepackage[english]{babel}
\usepackage{lipsum}
\usepackage{bm}
\usepackage{upgreek}
\usepackage{graphicx}

\usepackage{amsmath}
% mathtools for: Aboxed (put box on last equation in align envirenment)
\usepackage{microtype} %improves the spacing between words and letters
%% COLOR DEFINITIONS

%\usepackage[svgnames]{xcolor} % Enabling mixing colors and color's call by 'svgnames'
\usepackage{listings}
\usepackage[usenames,dvipsnames]{color}    

\lstset{ 
  language=R,                     % the language of the code
  basicstyle=\tiny\ttfamily, % the size of the fonts that are used for the code
  numbers=left,                   % where to put the line-numbers
  numberstyle=\tiny\color{Blue},  % the style that is used for the line-numbers
  stepnumber=1,                   % the step between two line-numbers. If it is 1, each line
                                  % will be numbered
  numbersep=5pt,                  % how far the line-numbers are from the code
  backgroundcolor=\color{white},  % choose the background color. You must add \usepackage{color}
  showspaces=false,               % show spaces adding particular underscores
  showstringspaces=false,         % underline spaces within strings
  showtabs=false,                 % show tabs within strings adding particular underscores
  frame=single,                   % adds a frame around the code
  rulecolor=\color{black},        % if not set, the frame-color may be changed on line-breaks within not-black text (e.g. commens (green here))
  tabsize=2,                      % sets default tabsize to 2 spaces
  captionpos=b,                   % sets the caption-position to bottom
  breaklines=true,                % sets automatic line breaking
  breakatwhitespace=false,        % sets if automatic breaks should only happen at whitespace
  keywordstyle=\color{RoyalBlue},      % keyword style
  commentstyle=\color{Green},   % comment style
  stringstyle=\color{Orange}      % string literal style
}  

\definecolor{MyColor1}{rgb}{0.2,0.4,0.6} %mix personal color
\newcommand{\textb}{\color{Black} \usefont{OT1}{lmss}{m}{n}}
\newcommand{\blue}{\color{MyColor1} \usefont{OT1}{lmss}{m}{n}}
\newcommand{\blueb}{\color{MyColor1} \usefont{OT1}{lmss}{b}{n}}
\newcommand{\red}{\color{LightCoral} \usefont{OT1}{lmss}{m}{n}}
\newcommand{\green}{\color{Turquoise} \usefont{OT1}{lmss}{m}{n}}

\DeclareMathOperator{\trace}{trace}
\DeclareMathOperator{\diag}{diag}

%% FONTS AND COLORS

%    SECTIONS

\usepackage{titlesec}
\usepackage{sectsty}
%%%%%%%%%%%%%%%%%%%%%%%%
%set section/subsections HEADINGS font and color
\sectionfont{\color{MyColor1}}  % sets colour of sections
\subsectionfont{\color{MyColor1}}  % sets colour of sections

%set section enumerator to arabic number (see footnotes markings alternatives)
\renewcommand\thesection{\arabic{section}.} %define sections numbering
\renewcommand\thesubsection{\thesection\arabic{subsection}} %subsec.num.

%define new section style
\newcommand{\mysection}{
\titleformat{\section} [runin] {\usefont{OT1}{lmss}{b}{n}\color{MyColor1}} 
{\thesection} {3pt} {} } 


%	CAPTIONS
\usepackage{caption}
\usepackage{subcaption}
%%%%%%%%%%%%%%%%%%%%%%%%
\captionsetup[figure]{labelfont={color=Turquoise}}


%		!!!EQUATION (ARRAY) --> USING ALIGN INSTEAD
%using amsmath package to redefine eq. numeration (1.1, 1.2, ...) 
\renewcommand{\theequation}{\thesection\arabic{equation}}



\makeatletter
\let\reftagform@=\tagform@
\def\tagform@#1{\maketag@@@{(\ignorespaces\textcolor{red}{#1}\unskip\@@italiccorr)}}
\renewcommand{\eqref}[1]{\textup{\reftagform@{\ref{#1}}}}
\makeatother
\usepackage{hyperref}
\hypersetup{colorlinks=true}

% For labeling top of page on every page but first one:
\usepackage{fancyhdr}

% PREPARE TITLE:
\title{\blue CSCI4113 \\
\blueb LAB5 B Notes}
\author{Milan Formanek}
\date{\today} % You can set the date automatically by replacing "date goes here" with "\today"

\renewcommand{\rmdefault}{phv} % Arial Font
\renewcommand{\sfdefault}{phv} % Arial Font

\pagestyle{fancy}
\fancyhead{}
\fancyhead[CO,CE]{{\small{{\bf{LAB5 B Notes}} - CSCI4113 - Spring 19 - Milan Formanek}}}



\begin{document}
\maketitle
\section{Configuring rsync between machines B and F}
Rsync has to be installed on both machines B and F:
\begin{lstlisting}
[root@carriage ~]# yum install rsync
\end{lstlisting}
Then the user group permissions and DM site are pulled from machine B to machine F:
\begin{lstlisting}
[root@saddle ~]# rsync -avz root@carriage:/etc/sudoers /etc/sudoers
[root@saddle ~]# rsync -avz root@carriage:/etc/shadow /etc/shadow
[root@saddle ~]# rsync -avz root@carriage:/etc/passwd /etc/passwd
[root@saddle ~]# rsync -avz root@carriage:/etc/group /etc/group
[root@saddle ~]# rsync -avz root@carriage:/var/www/ /var/www/
\end{lstlisting}
To schedule the DM site to get pulled from machine B to F a crontab rule is set up:
\begin{lstlisting}
[root@saddle ~]# nano /etc/crontab
File: /etc/crontab
SHELL=/bin/bash
PATH=/sbin:/bin:/usr/sbin:/usr/bin
MAILTO=root
# For details see man 4 crontabs
# Example of job definition:
# .---------------- minute (0 - 59)
# |  .------------- hour (0 - 23)
# |  |  .---------- day of month (1 - 31)
# |  |  |  .------- month (1 - 12) OR jan,feb,mar,apr ...
# |  |  |  |  .---- day of week (0 - 6) (Sunday=0 or 7) OR sun,mon,tue,wed,thu,$
# |  |  |  |  |
# *  *  *  *  * user-name  command to be executed
0 * * * * root rsync -avz root@carriage:/var/www/ /var/www/
[root@saddle ~]# systemctl restart crond
[root@saddle ~]# systemctl enable crond
\end{lstlisting}
The crond deamon is then restarted and enabled. 
\section{Configuring machine D as the DNS server}
To Configure Machine D as the DNS server Bind is installed with:
\begin{lstlisting}
[root@chase ~]# yum install bind bind-utils
\end{lstlisting}
After that 3 config files have to be set up:
\begin{lstlisting}
[root@chase ~]# nano /etc/named.conf
\end{lstlisting}
The named.conf file sets up the basic configuration for the Named deamon. It's important to set recursion to yes and set up the dundermifflin.com zone correctly.
\begin{lstlisting}
File: /etc/named.conf
options {
        directory "/var/named";
        recursion yes;
        listen-on {127.0.0.1; 100.64.21.0/24; 10.32.21.2/24;};
        allow-query{any;};
};

zone "dundermifflin.com" {
        type master;
        file "dundermifflin.com.";
};
[root@chase ~]# chmod 777  /etc/named.conf
\end{lstlisting}
Next is the dundermifflin.com. config for the dundermifflin zone:
\begin{lstlisting}
[root@chase ~]# nano /var/named/dundermifflin.com.
File: /var/named/dundermifflin.com.
$TTL 1H
; any time you make a change to the domain, bump the
; "serial" setting below. the format is easy:
; YYYYMMDDI, with the I being an iterator in case you
; make more than one change during any one day
@     IN SOA   chase    hostmaster (
                        200405191 ; serial
                        8H        ; refresh
                        4H        ; retry
                        4W        ; expire
                        1D )      ; minimum
; chase.dundermifflin.com serves this domain as both the
; name server (NS) and mail exchange (MX)
                NS      chase.dundermifflin.com.
;               MX      10 chase
; define domain functions with CNAMEs
dundermifflin.com               300     IN       CNAME   carriage.dundermifflin.com
www.dundermifflin.com.          300     IN       CNAME   carriage.dundermifflin.com
www2.dundermifflin.com.         300     IN       CNAME   saddle.dundermifflin.com.
ftp.dundermifflin.com.          300     IN       CNAME   platen.dundermifflin.com.
files.dundermifflin.com.        604800  IN       CNAME   roller.dundermifflin.com.

machinea.dundermifflin.com.     604800    IN       CNAME   router.dundermifflin.com.
machineb.dundermifflin.com.     604800    IN       CNAME   carriage.dundermifflin.com.
machinec.dundermifflin.com.     604800    IN       CNAME   platen.dundermifflin.com.
machined.dundermifflin.com.     604800    IN       CNAME   chase.dundermifflin.com.
machinee.dundermifflin.com.     604800    IN       CNAME   roller.dundermifflin.com.
machinef.dundermifflin.com.     604800    IN       CNAME   saddle.dundermifflin.com.


; just in case someone asks for localhost.schroder.net
localhost                     IN     A       127.0.0.1
; our hostnames, in alphabetical order
router.dundermifflin.com.     IN     A       100.64.0.21
carriage.dundermifflin.com.   IN     A       100.64.21.2
platen.dundermifflin.com.     IN     A       100.64.21.3
chase.dundermifflin.com.      IN     A       100.64.21.4
roller.dundermifflin.com.     IN     A       100.21.32.2
saddle.dundermifflin.com.     IN     A       100.64.21.5
[root@chase ~]# chmod 777 /var/named/dundermifflin.com.
\end{lstlisting}
This file contains the ip bindings for the local dundermifflin.com domain as well as set chase as the Name Server for the network. \\
Last is the named.ca file containing information on DNS servers resolving domains not dundermifflin.com. This file can be downloaded from \url{https://www.internic.net/domain/named.root}.
\begin{lstlisting}
[root@chase ~]# nano /var/named/named.ca
File: /var/named/named.ca
;       This file holds the information on root name servers needed to
;       initialize cache of Internet domain name servers
;       (e.g. reference this file in the "cache  .  <file>"
;       configuration file of BIND domain name servers).
;
;       This file is made available by InterNIC
;       under anonymous FTP as
;           file                /domain/named.cache
;           on server           FTP.INTERNIC.NET
;       -OR-                    RS.INTERNIC.NET
;
;       last update:     March 13, 2019
;       related version of root zone:     2019031302
;
; FORMERLY NS.INTERNIC.NET
;
.                        3600000      NS    A.ROOT-SERVERS.NET.
A.ROOT-SERVERS.NET.      3600000      A     198.41.0.4
A.ROOT-SERVERS.NET.      3600000      AAAA  2001:503:ba3e::2:30
;
; FORMERLY NS1.ISI.EDU
;
.                        3600000      NS    B.ROOT-SERVERS.NET.
B.ROOT-SERVERS.NET.      3600000      A     199.9.14.201
B.ROOT-SERVERS.NET.      3600000      AAAA  2001:500:200::b
;
; FORMERLY C.PSI.NET
;
.                        3600000      NS    C.ROOT-SERVERS.NET.
C.ROOT-SERVERS.NET.      3600000      A     192.33.4.12
C.ROOT-SERVERS.NET.      3600000      AAAA  2001:500:2::c
;
; FORMERLY TERP.UMD.EDU
;
.                        3600000      NS    D.ROOT-SERVERS.NET.
D.ROOT-SERVERS.NET.      3600000      A     199.7.91.13
D.ROOT-SERVERS.NET.      3600000      AAAA  2001:500:2d::d
;
; FORMERLY NS.NASA.GOV
;
.                        3600000      NS    E.ROOT-SERVERS.NET.
E.ROOT-SERVERS.NET.      3600000      A     192.203.230.10
E.ROOT-SERVERS.NET.      3600000      AAAA  2001:500:a8::e
;
; FORMERLY NS.ISC.ORG
;
.                        3600000      NS    F.ROOT-SERVERS.NET.
F.ROOT-SERVERS.NET.      3600000      A     192.5.5.241
F.ROOT-SERVERS.NET.      3600000      AAAA  2001:500:2f::f
;
; FORMERLY NS.NIC.DDN.MIL
;
.                        3600000      NS    G.ROOT-SERVERS.NET.
G.ROOT-SERVERS.NET.      3600000      A     192.112.36.4
G.ROOT-SERVERS.NET.      3600000      AAAA  2001:500:12::d0d
;
; FORMERLY AOS.ARL.ARMY.MIL
;
.                        3600000      NS    H.ROOT-SERVERS.NET.
H.ROOT-SERVERS.NET.      3600000      A     198.97.190.53
[root@chase ~]# chmod 777 /var/named/named.ca
\end{lstlisting}
After this is done the Named demon has to be started and enabled to run on boot:
\begin{lstlisting}
[root@chase ~]# systemctl start named
[root@chase ~]# systemctl enable named
\end{lstlisting}
\section{Configuring machine A the Router}
Machine A has to be configured to advertise Machine D as the DNS sever for the local network. This is done by adding the line:   option domain-name-servers 100.64.21.4;
To both of the subnets served by the router.
\begin{lstlisting}
[root@router ~]# nano  /etc/dhcp/dhcpd.conf
\end{lstlisting}
After this is complete the dhcpd deamon has to be restarted for the new configuration to take effect. \\
The DNS server is set statically in the router by removing the line PEERDNS=no and adding the line DNS1=100.64.21.4 the 3 network interface config files:
\begin{lstlisting}
[root@router ~]# nano  /etc/sysconfig/network-scripts/ifcfg-ens256
[root@router ~]# nano  /etc/sysconfig/network-scripts/ifcfg-ens224
[root@router ~]# nano  /etc/sysconfig/network-scripts/ifcfg-ens192
\end{lstlisting}
Note the university DNS server at 128.138.130.30 is still kept as a backup.
\section{Configuring machines B,C,D,E,F to accept the new DNS Server}
Machines B,C,D,E and F need to be configured to take the new address of the local DNS server from DHCP. This is done by  the removing the line PEERDNS=no and adding the lines DNS1="" and DNS2="" to the network interface config file on all the above mentioned machines:
\begin{lstlisting}
[root@router ~]# nano  /etc/sysconfig/network-scripts/ifcfg-ens192
\end{lstlisting}
 For some reason without forcing the 2 DNS servers to be epty the machines would revert to the University DNS after reboot. \\
To check if the setting stuck the resolv.conf file is checked and a a dig is performed on both an external and internal domain:
\begin{lstlisting}
[root@carriage ~]# nano /etc/resolv.conf
File: /etc/resolv.conf
; generated by /usr/sbin/dhclient-script
search dundermifflin.com.
nameserver 100.64.21.4
[root@carriage ~]# dig dundermifflin.com
; <<>> DiG 9.9.4-RedHat-9.9.4-73.el7_6 <<>> dundermifflin.com
;; global options: +cmd
;; Got answer:
;; ->>HEADER<<- opcode: QUERY, status: NOERROR, id: 39724
;; flags: qr aa rd ra; QUERY: 1, ANSWER: 0, AUTHORITY: 1, ADDITIONAL: 1
;; OPT PSEUDOSECTION:
; EDNS: version: 0, flags:; udp: 4096
;; QUESTION SECTION:
;dundermifflin.com.             IN      A
;; AUTHORITY SECTION:
dundermifflin.com.      3600    IN      SOA     chase.dundermifflin.com. hostmaster.dundermifflin.com. 200405191 28800 14400 2419200 86400
;; Query time: 1 msec
;; SERVER: 100.64.21.4#53(100.64.21.4)
;; WHEN: Tue Apr 09 01:00:44 MDT 2019
;; MSG SIZE  rcvd: 99
[root@carriage ~]# dig www.google.com
; <<>> DiG 9.9.4-RedHat-9.9.4-73.el7_6 <<>> www.google.com
;; global options: +cmd
;; Got answer:
;; ->>HEADER<<- opcode: QUERY, status: NOERROR, id: 14187
;; flags: qr rd ra; QUERY: 1, ANSWER: 1, AUTHORITY: 4, ADDITIONAL: 9
;; OPT PSEUDOSECTION:
; EDNS: version: 0, flags:; udp: 4096
;; QUESTION SECTION:
;www.google.com.                        IN      A
;; ANSWER SECTION:
www.google.com.         300     IN      A       172.217.2.4
;; AUTHORITY SECTION:
google.com.             171975  IN      NS      ns3.google.com.
google.com.             171975  IN      NS      ns2.google.com.
google.com.             171975  IN      NS      ns4.google.com.
google.com.             171975  IN      NS      ns1.google.com.
;; ADDITIONAL SECTION:
ns2.google.com.         171975  IN      A       216.239.34.10
ns2.google.com.         171975  IN      AAAA    2001:4860:4802:34::a
ns1.google.com.         171975  IN      A       216.239.32.10
ns1.google.com.         171975  IN      AAAA    2001:4860:4802:32::a
ns3.google.com.         171975  IN      A       216.239.36.10
ns3.google.com.         171975  IN      AAAA    2001:4860:4802:36::a
ns4.google.com.         171975  IN      A       216.239.38.10
ns4.google.com.         171975  IN      AAAA    2001:4860:4802:38::a
;; Query time: 26 msec
;; SERVER: 100.64.21.4#53(100.64.21.4)
;; WHEN: Tue Apr 09 01:01:52 MDT 2019
;; MSG SIZE  rcvd: 307
\end{lstlisting}
Towards the end of the Dig output on the SERVER: line you can see the domains resolve to our local DNS server at 100.64.21.4.
\end{document}
