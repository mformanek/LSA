\documentclass[11pt,onside]{article}
\usepackage[a4paper]{geometry}
\usepackage[utf8]{inputenc}
\usepackage[english]{babel}
\usepackage{lipsum}
\usepackage{bm}
\usepackage{upgreek}
\usepackage{graphicx}

\usepackage{amsmath}
% mathtools for: Aboxed (put box on last equation in align envirenment)
\usepackage{microtype} %improves the spacing between words and letters
%% COLOR DEFINITIONS

%\usepackage[svgnames]{xcolor} % Enabling mixing colors and color's call by 'svgnames'
\usepackage{listings}
\usepackage[usenames,dvipsnames]{color}    

\lstset{ 
  language=R,                     % the language of the code
  basicstyle=\tiny\ttfamily, % the size of the fonts that are used for the code
  numbers=left,                   % where to put the line-numbers
  numberstyle=\tiny\color{Blue},  % the style that is used for the line-numbers
  stepnumber=1,                   % the step between two line-numbers. If it is 1, each line
                                  % will be numbered
  numbersep=5pt,                  % how far the line-numbers are from the code
  backgroundcolor=\color{white},  % choose the background color. You must add \usepackage{color}
  showspaces=false,               % show spaces adding particular underscores
  showstringspaces=false,         % underline spaces within strings
  showtabs=false,                 % show tabs within strings adding particular underscores
  frame=single,                   % adds a frame around the code
  rulecolor=\color{black},        % if not set, the frame-color may be changed on line-breaks within not-black text (e.g. commens (green here))
  tabsize=2,                      % sets default tabsize to 2 spaces
  captionpos=b,                   % sets the caption-position to bottom
  breaklines=true,                % sets automatic line breaking
  breakatwhitespace=false,        % sets if automatic breaks should only happen at whitespace
  keywordstyle=\color{RoyalBlue},      % keyword style
  commentstyle=\color{Green},   % comment style
  stringstyle=\color{Orange}      % string literal style
}  

\definecolor{MyColor1}{rgb}{0.2,0.4,0.6} %mix personal color
\newcommand{\textb}{\color{Black} \usefont{OT1}{lmss}{m}{n}}
\newcommand{\blue}{\color{MyColor1} \usefont{OT1}{lmss}{m}{n}}
\newcommand{\blueb}{\color{MyColor1} \usefont{OT1}{lmss}{b}{n}}
\newcommand{\red}{\color{LightCoral} \usefont{OT1}{lmss}{m}{n}}
\newcommand{\green}{\color{Turquoise} \usefont{OT1}{lmss}{m}{n}}

\DeclareMathOperator{\trace}{trace}
\DeclareMathOperator{\diag}{diag}

%% FONTS AND COLORS

%    SECTIONS

\usepackage{titlesec}
\usepackage{sectsty}
%%%%%%%%%%%%%%%%%%%%%%%%
%set section/subsections HEADINGS font and color
\sectionfont{\color{MyColor1}}  % sets colour of sections
\subsectionfont{\color{MyColor1}}  % sets colour of sections

%set section enumerator to arabic number (see footnotes markings alternatives)
\renewcommand\thesection{\arabic{section}.} %define sections numbering
\renewcommand\thesubsection{\thesection\arabic{subsection}} %subsec.num.

%define new section style
\newcommand{\mysection}{
\titleformat{\section} [runin] {\usefont{OT1}{lmss}{b}{n}\color{MyColor1}} 
{\thesection} {3pt} {} } 


%	CAPTIONS
\usepackage{caption}
\usepackage{subcaption}
%%%%%%%%%%%%%%%%%%%%%%%%
\captionsetup[figure]{labelfont={color=Turquoise}}


%		!!!EQUATION (ARRAY) --> USING ALIGN INSTEAD
%using amsmath package to redefine eq. numeration (1.1, 1.2, ...) 
\renewcommand{\theequation}{\thesection\arabic{equation}}



\makeatletter
\let\reftagform@=\tagform@
\def\tagform@#1{\maketag@@@{(\ignorespaces\textcolor{red}{#1}\unskip\@@italiccorr)}}
\renewcommand{\eqref}[1]{\textup{\reftagform@{\ref{#1}}}}
\makeatother
\usepackage{hyperref}
\hypersetup{colorlinks=true}

% For labeling top of page on every page but first one:
\usepackage{fancyhdr}

% PREPARE TITLE:
\title{\blue CSCI4113 \\
\blueb LAB6 Notes}
\author{Milan Formanek}
\date{\today} % You can set the date automatically by replacing "date goes here" with "\today"

\renewcommand{\rmdefault}{phv} % Arial Font
\renewcommand{\sfdefault}{phv} % Arial Font

\pagestyle{fancy}
\fancyhead{}
\fancyhead[CO,CE]{{\small{{\bf{LAB6 Notes}} - CSCI4113 - Spring 19 - Milan Formanek}}}



\begin{document}
\maketitle
\section{Configuring IPtables firewall settings for the DM network}
For this lab the DunderMifflin network gets it's firewalls configured. Each of the machines A through F have to have unique IPtables rules based on their uses and running services. \\
To make the deployment easy a .sh script is created, grabbed from github and run with the right parameters for each machine.

\section{Configuring Machine B and F - Web Servers}
Machines B and F have identical configuration allowing http and https traffic incoming from any ip along with local loopback, icmp traffic and ssh from the local networks. This is done by deploying the LAB6.sh script to the machine with arguments B and F:
\begin{lstlisting}
[root@carriage ~]# wget https://raw.githubusercontent.com/mformanek/Linux-Sys-Admin---LAB6/master/LAB6.sh                --2019-04-21 23:20:09--  https://raw.githubusercontent.com/mformanek/Linux-Sys-Admin---LAB6/master/LAB6.sh
Resolving raw.githubusercontent.com (raw.githubusercontent.com)... 151.101.68.133
Connecting to raw.githubusercontent.com (raw.githubusercontent.com)|151.101.68.133|:443... connected.
HTTP request sent, awaiting response... 200 OK
Length: 6233 (6.1K) [text/plain]
Saving to: ‘LAB6.sh’
100%[===============================================================================>] 6,233       --.-K/s   in 0s
2019-04-21 23:20:09 (29.4 MB/s) - ‘LAB6.sh’ saved [6233/6233]
[root@carriage ~]# bash ./LAB6.sh B
iptables: Saving firewall rules to /etc/sysconfig/iptables:[  OK  ]
\end{lstlisting}
\begin{lstlisting}
[root@saddle ~]# wget https://raw.githubusercontent.com/mformanek/Linux-Sys-Admin---LAB6/master/LAB6.sh
--2019-04-21 23:25:59--  https://raw.githubusercontent.com/mformanek/Linux-Sys-Admin---LAB6/master/LAB6.sh
Resolving raw.githubusercontent.com (raw.githubusercontent.com)... 151.101.68.133
Connecting to raw.githubusercontent.com (raw.githubusercontent.com)|151.101.68.133|:443... connected.
HTTP request sent, awaiting response... 200 OK
Length: 6233 (6.1K) [text/plain]
Saving to: ‘LAB6.sh’
100%[============================================================================>] 6,233       --.-K/s   in 0s
2019-04-21 23:25:59 (22.9 MB/s) - ‘LAB6.sh’ saved [6233/6233]
[root@saddle ~]# bash ./LAB6.sh F
iptables: Saving firewall rules to /etc/sysconfig/iptables:[  OK  ]
\end{lstlisting}
\clearpage{}
\section{Configuring Machine C - FTP Server}
Machine C allows FTP connections from the 100.64.0/16 subnet along with local loopback, icmp traffic, DNS requests from Machine D, http and https outbound traffic and ssh from the local networks. This is done by deploying the LAB6.sh script to the machine with argument C:
\begin{lstlisting}
[root@platen ~]# wget https://raw.githubusercontent.com/mformanek/Linux-Sys-Admin---LAB6/master/LAB6.sh
--2019-04-22 00:32:11--  https://raw.githubusercontent.com/mformanek/Linux-Sys-Admin---LAB6/master/LAB6.sh
Resolving raw.githubusercontent.com (raw.githubusercontent.com)... 151.101.68.133
Connecting to raw.githubusercontent.com (raw.githubusercontent.com)|151.101.68.133|:443... connected.
HTTP request sent, awaiting response... 200 OK
Length: 6233 (6.1K) [text/plain]
Saving to: ‘LAB6.sh’
100%[============================================================================>] 6,233       --.-K/s   in 0s
2019-04-22 00:32:11 (26.6 MB/s) - ‘LAB6.sh’ saved [6233/6233]
[root@platen ~]# bash ./LAB6.sh C
iptables: Saving firewall rules to /etc/sysconfig/iptables:[  OK  ]
\end{lstlisting}
\section{Configuring Machine D - DNS Server}
Machine D allows DNS queries from any source along with local loopback, icmp traffic and ssh from the local networks. This is done by deploying the LAB6.sh script to the machine with argument D:
\begin{lstlisting}
[root@chase ~]# wget https://raw.githubusercontent.com/mformanek/Linux-Sys-Admin---LAB6/master/LAB6.sh
--2019-04-21 23:32:30--  https://raw.githubusercontent.com/mformanek/Linux-Sys-Admin---LAB6/master/LAB6.sh
Resolving raw.githubusercontent.com (raw.githubusercontent.com)... 151.101.68.133
Connecting to raw.githubusercontent.com (raw.githubusercontent.com)|151.101.68.133|:443... connected.
HTTP request sent, awaiting response... 200 OK
Length: 6233 (6.1K) [text/plain]
Saving to: ‘LAB6.sh’

100%[=========================================================>] 6,233       --.-K/s   in 0s

2019-04-21 23:32:30 (25.0 MB/s) - ‘LAB6.sh’ saved [6233/6233]

[root@chase ~]# bash ./LAB6.sh D
iptables: Saving firewall rules to /etc/sysconfig/iptables:[  OK  ]
\end{lstlisting}
\section{Configuring Machine E - File Server}
Machine E allows for Samba connections from the 10.21.32.0/24 subnet along with local loopback, icmp traffic and ssh from the same subnet. This is done by deploying the LAB6.sh script to the machine with argument E:
\begin{lstlisting}
[root@roller ~]# wget https://raw.githubusercontent.com/mformanek/Linux-Sys-Admin---LAB6/master/LAB6.sh
--2019-04-21 23:47:44--  https://raw.githubusercontent.com/mformanek/Linux-Sys-Admin---LAB6/master/LAB6.sh
Resolving raw.githubusercontent.com (raw.githubusercontent.com)... 151.101.68.133
Connecting to raw.githubusercontent.com (raw.githubusercontent.com)|151.101.68.133|:443... connected.
HTTP request sent, awaiting response... 200 OK
Length: 6233 (6.1K) [text/plain]
Saving to: ‘LAB6.sh’

100%[==========================================================================>] 6,233       --.-K/s   in 0s
2019-04-21 23:47:44 (22.5 MB/s) - ‘LAB6.sh’ saved [6233/6233]
[root@roller ~]# bash ./LAB6.sh E
iptables: Saving firewall rules to /etc/sysconfig/iptables:[  OK  ]
\end{lstlisting}
\section{Configuring Machine A - Router}
Finally the most complicated one, Machine A. It has to forward packets to the other computers in the DM network. It also mirrors the firewall settings on the individual machines in order to provide an extra layer of security on the network. Rules to block Facebook.com,icanhas.cheezburger.com and cheezburger.com are also implemented here.  Icanhas.cheezburger.com and cheezburger.com have the same IP making life a little easier.
\begin{lstlisting}
[root@router ~]# wget https://raw.githubusercontent.com/mformanek/Linux-Sys-Admin---LAB6/master/LAB6.sh
--2019-04-22 18:36:56--  https://raw.githubusercontent.com/mformanek/Linux-Sys-Admin---LAB6/master/LAB6.sh
Resolving raw.githubusercontent.com (raw.githubusercontent.com)... 151.101.68.133
Connecting to raw.githubusercontent.com (raw.githubusercontent.com)|151.101.68.133|:443... connected.
HTTP request sent, awaiting response... 200 OK
Length: 6136 (6.0K) [text/plain]
Saving to: ‘LAB6.sh’
100%[=============================================================================================================>] 6,136       --.-K/s   in 0s
2019-04-22 18:36:56 (19.2 MB/s) - ‘LAB6.sh’ saved [6136/6136]
[root@router ~]# bash ./LAB6.sh A
iptables: Saving firewall rules to /etc/sysconfig/iptables:[  OK  ]
\end{lstlisting}
\section{LAB6.sh - Deployment Script}
\begin{lstlisting}
#!/bin/bash
# ------------------------------------------------------------------
# By Milan Formanek 	LAB6 Deployment Script
# ------------------------------------------------------------------
VERSION=0.1.0
SUBJECT=some-unique-id
USAGE="Run on the individual DM machines with the letter name of the machine as the parameter."

if [ $# == 0 ] ; then
    echo $USAGE
    exit 1;
fi

if [ $1 == "OFF" ] ; then #ENABLE EVERITHING IN IPTABLES for testing
    iptables -P INPUT ACCEPT 
    iptables -P FORWARD ACCEPT 
    iptables -P OUTPUT ACCEPT
    iptables -F #reset IPTABLES
    service iptables save # make sure to save rules!!! 
    exit 1;
fi

iptables -F #reset IPTABLES

iptables -A INPUT -m conntrack --ctstate ESTABLISHED,RELATED -j ACCEPT #Allow return packets for ESTABLISHED and RELATED packets

iptables -P INPUT DROP # set default DROP policy 
iptables -P OUTPUT ACCEPT

iptables -A INPUT -i lo -j ACCEPT  #Allow loopback traffic
iptables -A OUTPUT -o lo -j ACCEPT

iptables -A INPUT -p icmp --icmp-type echo-request -j ACCEPT
iptables -A INPUT -p icmp --icmp-type echo-reply -j ACCEPT
iptables -A INPUT -p icmp --icmp-type time-exceeded -j ACCEPT
iptables -A INPUT -p icmp --icmp-type destination-unreachable -j ACCEPT #ACCEPT ICMP packets.

if [ $1 != "E" ] ; then
	iptables -A INPUT -p tcp -s  100.64.0.0/16 --dport 22 -m state --state NEW,ESTABLISHED -j ACCEPT
	iptables -A INPUT -p tcp -s  10.21.32.0/24 --dport 22 -m state --state NEW,ESTABLISHED -j ACCEPT
	iptables -A INPUT -p tcp -s  198.18.0.0/16 --dport 22 -m state --state NEW,ESTABLISHED -j ACCEPT
	iptables -A OUTPUT -p tcp --sport 22 -m state --state ESTABLISHED -j ACCEPT
	#On all machines exlcuding E allow inbound ssh connections from the 100.64.0.0/16, 10.21.32.0/24, and 198.18.0.0/16 subnets
fi

if [ $1 != "A" ] ; then
	iptables -P FORWARD DROP #disble forwarding on non routers
else #RULES FOR ROUTER/MACHINE A
    	iptables -P FORWARD DROP #enable forwarding on routers
    	iptables -A FORWARD -s 157.240.28.35 -j DROP
    	iptables -A FORWARD -d 157.240.28.35 -j DROP #block FACEBOOK
    	iptables -A FORWARD -s 216.176.177.74 -j DROP
    	iptables -A FORWARD -d 216.176.177.74 -j DROP #block CHEESEBURGER.com
    
    	iptables -A FORWARD -p tcp -s  100.64.0.0/16 --dport 22 -m state --state NEW,ESTABLISHED -j ACCEPT
	iptables -A FORWARD -p tcp -s  10.21.32.0/24 --dport 22 -m state --state NEW,ESTABLISHED -j ACCEPT
	iptables -A FORWARD -p tcp -s  198.18.0.0/16 --dport 22 -m state --state NEW,ESTABLISHED -j ACCEPT
    	iptables -A FORWARD -m state --state ESTABLISHED,RELATED -j ACCEPT #FORWARD SSH
	
	iptables -A FORWARD -p tcp --dport 80   -m state --state NEW,ESTABLISHED,RELATED -j ACCEPT
	iptables -A FORWARD -p tcp --dport 443  -m state --state NEW,ESTABLISHED,RELATED -j ACCEPT 
	
	iptables -A FORWARD -p icmp --icmp-type echo-request -j ACCEPT
	iptables -A FORWARD -p icmp --icmp-type echo-reply -j ACCEPT
	iptables -A FORWARD -p icmp --icmp-type time-exceeded -j ACCEPT
	iptables -A FORWARD -p icmp --icmp-type destination-unreachable -j ACCEPT #ACCEPT ICMP packets
	

	iptables -A FORWARD -p udp --sport 1024:65535 --dport 53 -m state --state NEW,ESTABLISHED -j ACCEPT
	iptables -A FORWARD -p udp --sport 53 --dport 1024:65535 -m state --state ESTABLISHED -j ACCEPT
	iptables -A FORWARD -p udp --sport 53 --dport 53 -m state --state NEW,ESTABLISHED -j ACCEPT
	iptables -A FORWARD -p udp --sport 53 --dport 53 -m state --state ESTABLISHED -j ACCEPT 
	#allow inbound DNS lookup on chase
	
	

	iptables -A FORWARD -m conntrack --ctstate ESTABLISHED,RELATED -j ACCEPT #Allow return packets for ESTABLISHED and RELATED packets
	

fi

if [ $1 == "B" ] || [ $1 == "F" ] ; then #RULES FOR MACHINE B AND F
	iptables -A INPUT -p tcp -m tcp --dport 80 -j ACCEPT
	iptables -A INPUT -p tcp -m tcp --dport 443 -j ACCEPT #allow http and https inbound traffic
fi

if [ $1 == "C" ] ; then #RULES FOR MACHINE C
	iptables -P OUTPUT DROP #defaul output drop
	
	iptables -A INPUT -m state --state ESTABLISHED,RELATED -j ACCEPT
	iptables -A OUTPUT -m state --state ESTABLISHED,RELATED -j ACCEPT #allow related connections
	
	iptables -A OUTPUT -p udp -d 100.64.21.4 --dport 53 -m state --state NEW,ESTABLISHED -j ACCEPT
	iptables -A INPUT  -p udp -s 100.64.21.4 --sport 53 -m state --state ESTABLISHED     -j ACCEPT
	iptables -A OUTPUT -p tcp -d 100.64.21.4 --dport 53 -m state --state NEW,ESTABLISHED -j ACCEPT
	iptables -A INPUT  -p tcp -s 100.64.21.4 --sport 53 -m state --state ESTABLISHED     -j ACCEPT #allow DNS lookup on chase
	
	iptables -A OUTPUT -p tcp -m tcp --dport 80 -j ACCEPT
	iptables -A OUTPUT -p tcp -m tcp --dport 443 -j ACCEPT #allow outbound http and https traffic
	
	iptables -A OUTPUT -p tcp --dport 22 -j ACCEPT #allow outgoing ssh
	
	iptables -A OUTPUT -p icmp --icmp-type echo-request -j ACCEPT
	iptables -A OUTPUT -p icmp --icmp-type echo-reply -j ACCEPT
	iptables -A OUTPUT -p icmp --icmp-type time-exceeded -j ACCEPT
	iptables -A OUTPUT -p icmp --icmp-type destination-unreachable -j ACCEPT #ACCEPT outbound ICMP packets #allow outgoing ICMP
	
	iptables -A OUTPUT -p tcp --sport 21 -m state --state ESTABLISHED -j ACCEPT
	iptables -A OUTPUT -p tcp --sport 20 -m state --state ESTABLISHED,RELATED -j ACCEPT
	iptables -A OUTPUT -p tcp --sport 1024: --dport 1024: -m state --state ESTABLISHED -j ACCEPT
	iptables -A INPUT -p tcp --dport 21 -m state --state NEW,ESTABLISHED -j ACCEPT
	iptables -A INPUT -p tcp --dport 20 -m state --state ESTABLISHED -j ACCEPT
	iptables -A INPUT -p tcp --sport 1024: --dport 1024: -m state --state ESTABLISHED,RELATED,NEW -j ACCEPT
	#FTP Rules
fi

if [ $1 == "D" ] ; then #RULES FOR MACHINE D - DNS SERVER
	SERVER_IP="100.64.21.4"
	iptables -A INPUT -p udp -s 0/0 --sport 1024:65535 -d $SERVER_IP --dport 53 -m state --state NEW,ESTABLISHED -j ACCEPT
	iptables -A OUTPUT -p udp -s $SERVER_IP --sport 53 -d 0/0 --dport 1024:65535 -m state --state ESTABLISHED -j ACCEPT
	iptables -A INPUT -p udp -s 0/0 --sport 53 -d $SERVER_IP --dport 53 -m state --state NEW,ESTABLISHED -j ACCEPT
	iptables -A OUTPUT -p udp -s $SERVER_IP --sport 53 -d 0/0 --dport 53 -m state --state ESTABLISHED -j ACCEPT 
	#allow inbound DNS lookup on chase.
fi

if [ $1 == "E" ] ; then #RULES FOR MACHINE E - FILE SERVER
	iptables -A INPUT -p tcp -s 10.21.32.0/24 --dport 22 -m state --state NEW,ESTABLISHED -j ACCEPT
	iptables -A OUTPUT -p tcp --sport 22 -m state --state ESTABLISHED -j ACCEPT #enable SSH connections only from 10.21.32.0/24 net
	
	iptables -A INPUT -m state --state NEW -p udp --dport 137 -j ACCEPT
	iptables -A INPUT -m state --state NEW -p udp --dport 138 -j ACCEPT
	iptables -A INPUT -m state --state NEW -p tcp --dport 139 -j ACCEPT
	iptables -A INPUT -m state --state NEW -p tcp --dport 445 -j ACCEPT
	#allow incoming connections for CIFS and SMB
fi

service iptables save # make sure to save rules!!! 
\end{lstlisting}

\end{document}
