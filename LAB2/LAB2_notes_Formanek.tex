\documentclass[11pt,onside]{article}
\usepackage[a4paper]{geometry}
\usepackage[utf8]{inputenc}
\usepackage[english]{babel}
\usepackage{lipsum}
\usepackage{bm}
\usepackage{upgreek}
\usepackage{graphicx}

\usepackage{amsmath}
% mathtools for: Aboxed (put box on last equation in align envirenment)
\usepackage{microtype} %improves the spacing between words and letters
%% COLOR DEFINITIONS

%\usepackage[svgnames]{xcolor} % Enabling mixing colors and color's call by 'svgnames'
\usepackage{listings}
\usepackage[usenames,dvipsnames]{color}    

\lstset{ 
  language=R,                     % the language of the code
  basicstyle=\tiny\ttfamily, % the size of the fonts that are used for the code
  numbers=left,                   % where to put the line-numbers
  numberstyle=\tiny\color{Blue},  % the style that is used for the line-numbers
  stepnumber=1,                   % the step between two line-numbers. If it is 1, each line
                                  % will be numbered
  numbersep=5pt,                  % how far the line-numbers are from the code
  backgroundcolor=\color{white},  % choose the background color. You must add \usepackage{color}
  showspaces=false,               % show spaces adding particular underscores
  showstringspaces=false,         % underline spaces within strings
  showtabs=false,                 % show tabs within strings adding particular underscores
  frame=single,                   % adds a frame around the code
  rulecolor=\color{black},        % if not set, the frame-color may be changed on line-breaks within not-black text (e.g. commens (green here))
  tabsize=2,                      % sets default tabsize to 2 spaces
  captionpos=b,                   % sets the caption-position to bottom
  breaklines=true,                % sets automatic line breaking
  breakatwhitespace=false,        % sets if automatic breaks should only happen at whitespace
  keywordstyle=\color{RoyalBlue},      % keyword style
  commentstyle=\color{Green},   % comment style
  stringstyle=\color{Orange}      % string literal style
}  

\definecolor{MyColor1}{rgb}{0.2,0.4,0.6} %mix personal color
\newcommand{\textb}{\color{Black} \usefont{OT1}{lmss}{m}{n}}
\newcommand{\blue}{\color{MyColor1} \usefont{OT1}{lmss}{m}{n}}
\newcommand{\blueb}{\color{MyColor1} \usefont{OT1}{lmss}{b}{n}}
\newcommand{\red}{\color{LightCoral} \usefont{OT1}{lmss}{m}{n}}
\newcommand{\green}{\color{Turquoise} \usefont{OT1}{lmss}{m}{n}}

\DeclareMathOperator{\trace}{trace}
\DeclareMathOperator{\diag}{diag}

%% FONTS AND COLORS

%    SECTIONS

\usepackage{titlesec}
\usepackage{sectsty}
%%%%%%%%%%%%%%%%%%%%%%%%
%set section/subsections HEADINGS font and color
\sectionfont{\color{MyColor1}}  % sets colour of sections
\subsectionfont{\color{MyColor1}}  % sets colour of sections

%set section enumerator to arabic number (see footnotes markings alternatives)
\renewcommand\thesection{\arabic{section}.} %define sections numbering
\renewcommand\thesubsection{\thesection\arabic{subsection}} %subsec.num.

%define new section style
\newcommand{\mysection}{
\titleformat{\section} [runin] {\usefont{OT1}{lmss}{b}{n}\color{MyColor1}} 
{\thesection} {3pt} {} } 


%	CAPTIONS
\usepackage{caption}
\usepackage{subcaption}
%%%%%%%%%%%%%%%%%%%%%%%%
\captionsetup[figure]{labelfont={color=Turquoise}}


%		!!!EQUATION (ARRAY) --> USING ALIGN INSTEAD
%using amsmath package to redefine eq. numeration (1.1, 1.2, ...) 
\renewcommand{\theequation}{\thesection\arabic{equation}}



\makeatletter
\let\reftagform@=\tagform@
\def\tagform@#1{\maketag@@@{(\ignorespaces\textcolor{red}{#1}\unskip\@@italiccorr)}}
\renewcommand{\eqref}[1]{\textup{\reftagform@{\ref{#1}}}}
\makeatother
\usepackage{hyperref}
\hypersetup{colorlinks=true}

% For labeling top of page on every page but first one:
\usepackage{fancyhdr}

% PREPARE TITLE:
\title{\blue CSCI4113 \\
\blueb LAB2 Notes}
\author{Milan Formanek}
\date{\today} % You can set the date automatically by replacing "date goes here" with "\today"

\renewcommand{\rmdefault}{phv} % Arial Font
\renewcommand{\sfdefault}{phv} % Arial Font

\pagestyle{fancy}
\fancyhead{}
\fancyhead[CO,CE]{{\small{{\bf{LAB2 Notes}} - CSCI4113 - Spring 19 - Milan Formanek}}}



\begin{document}
\maketitle

\section{}
With the lsblk command you can list all the block devices.  
\begin{lstlisting}
[root@machinee ~]# lsblk
NAME            MAJ:MIN RM  SIZE RO TYPE MOUNTPOINT
fd0               2:0    1    4K  0 disk
sda               8:0    0    6G  0 disk
├─sda1            8:1    0  500M  0 part /boot
└─sda2            8:2    0  5.5G  0 part
  ├─centos-swap 253:0    0  512M  0 lvm  [SWAP]
  └─centos-root 253:1    0    5G  0 lvm  /
sdb               8:16   0  5.7G  0 disk
sr0              11:0    1 1024M  0 rom
[root@machinee ~]#
\end{lstlisting}
In this lab I will be adding sdb to the centos lvgroup, creating /home and /temp lvs. Migrating the data over to the new lvs from existing folders. Finally modifying /etc/fstab to auto mount the filesystems at desired locations.
\section{}

First off we will initialize /dev/sdb as a physical volume. 

\begin{lstlisting}
[root@machinee ~]# pvcreate /dev/sdb
  Physical volume "/dev/sdb" successfully created.
\end{lstlisting}

Then add the physical volume to the "centos" volume group.

\begin{lstlisting}
[root@machinee ~]# vgextend centos /dev/sdb
  Volume group "centos" successfully extended
\end{lstlisting}
Now we will create the 2 logical volumes home and tmp.
\begin{lstlisting}
[root@machinee ~]# lvcreate -n home -L 4G centos
  Logical volume "home" created.
[root@machinee ~]# lvcreate -n tmp -l 100%FREE centos
  Logical volume "tmp" created.
\end{lstlisting}
\clearpage()
We can see the new lvs are now created on /dev/sdb.
\begin{lstlisting}
[root@machinee ~]# lsblk
NAME            MAJ:MIN RM  SIZE RO TYPE MOUNTPOINT
fd0               2:0    1    4K  0 disk
sda               8:0    0    6G  0 disk
ââsda1            8:1    0  500M  0 part /boot
ââsda2            8:2    0  5.5G  0 part
  ââcentos-swap 253:0    0  512M  0 lvm  [SWAP]
  ââcentos-root 253:1    0    5G  0 lvm  /
sdb               8:16   0  5.7G  0 disk
ââcentos-home   253:2    0    4G  0 lvm
ââcentos-tmp    253:3    0  1.7G  0 lvm
sr0              11:0    1 1024M  0 rom
[root@machinee ~]#
\end{lstlisting}
\begin{lstlisting}
Last step in preparing the new storage space is to format the tmp and boot lvs.
[root@machinee ~]# mkfs.xfs /dev/centos/home
meta-data=/dev/centos/home       isize=512    agcount=4, agsize=262144 blks
         =                       sectsz=512   attr=2, projid32bit=1
         =                       crc=1        finobt=0, sparse=0
data     =                       bsize=4096   blocks=1048576, imaxpct=25
         =                       sunit=0      swidth=0 blks
naming   =version 2              bsize=4096   ascii-ci=0 ftype=1
log      =internal log           bsize=4096   blocks=2560, version=2
         =                       sectsz=512   sunit=0 blks, lazy-count=1
realtime =none                   extsz=4096   blocks=0, rtextents=0
[root@machinee ~]# mkfs.xfs /dev/centos/tmp
meta-data=/dev/centos/tmp        isize=512    agcount=4, agsize=112640 blks
         =                       sectsz=512   attr=2, projid32bit=1
         =                       crc=1        finobt=0, sparse=0
data     =                       bsize=4096   blocks=450560, imaxpct=25
         =                       sunit=0      swidth=0 blks
naming   =version 2              bsize=4096   ascii-ci=0 ftype=1
log      =internal log           bsize=4096   blocks=2560, version=2
         =                       sectsz=512   sunit=0 blks, lazy-count=1
realtime =none                   extsz=4096   blocks=0, rtextents=0
[root@machinee ~]#
\end{lstlisting}
Then we mount the new filesystems as tmp2 and home2 directories.
\begin{lstlisting}
[root@machinee ~]# cd /
[root@machinee /]# ls
bin   dev  home   lib    media  opt   root  sbin  sys  usr
boot  etc  home2  lib64  mnt    proc  run   srv   tmp  var
[root@machinee /]# mkdir tmp2
[root@machinee /]# mount /dev/centos/home /home2
[root@machinee /]# mount /dev/centos/home /tmp2
\end{lstlisting}
After that we move the contents of tmp and home to the new locations.
\begin{lstlisting}
[root@machinee ~]# mv /home/* /home2 
[root@machinee /]# mv /tmp/* /tmp2 
\end{lstlisting}
Then I unmount the temp mounts and reboot
\begin{lstlisting}
[root@machinee ~]# umount /home2
[root@machinee /]# umount /tmp2
[root@machinee /]# reboot
\end{lstlisting}
\section{}
Last step is modifying the /etc/fstab config file.
\begin{lstlisting}
[root@machinee ~]# nano nano /etc/fstab
\end{lstlisting}
Then finally the fstab file is edited to represent the table in the lab2 guide.

\end{document}